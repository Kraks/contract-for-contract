%% For double-blind review submission, w/o CCS and ACM Reference (max submission space)
\documentclass[acmsmall,review,anonymous]{acmart}\settopmatter{printfolios=true,printccs=false,printacmref=false}
%% For double-blind review submission, w/ CCS and ACM Reference
%\documentclass[acmsmall,review,anonymous]{acmart}\settopmatter{printfolios=true}
%% For single-blind review submission, w/o CCS and ACM Reference (max submission space)
%\documentclass[acmsmall,review]{acmart}\settopmatter{printfolios=true,printccs=false,printacmref=false}
%% For single-blind review submission, w/ CCS and ACM Reference
%\documentclass[acmsmall,review]{acmart}\settopmatter{printfolios=true}
%% For final camera-ready submission, w/ required CCS and ACM Reference
%\documentclass[acmsmall]{acmart}\settopmatter{}


%% Journal information
%% Supplied to authors by publisher for camera-ready submission;
%% use defaults for review submission.
\acmJournal{PACMPL}
\acmVolume{1}
\acmNumber{CONF} % CONF = POPL or ICFP or OOPSLA
\acmArticle{1}
\acmYear{2018}
\acmMonth{1}
\acmDOI{} % \acmDOI{10.1145/nnnnnnn.nnnnnnn}
\startPage{1}

%% Copyright information
%% Supplied to authors (based on authors' rights management selection;
%% see authors.acm.org) by publisher for camera-ready submission;
%% use 'none' for review submission.
\setcopyright{none}
%\setcopyright{acmcopyright}
%\setcopyright{acmlicensed}
%\setcopyright{rightsretained}
%\copyrightyear{2018}           %% If different from \acmYear

%% Bibliography style
\bibliographystyle{ACM-Reference-Format}
%% Citation style
%% Note: author/year citations are required for papers published as an
%% issue of PACMPL.
\citestyle{acmauthoryear}   %% For author/year citations


%%%%%%%%%%%%%%%%%%%%%%%%%%%%%%%%%%%%%%%%%%%%%%%%%%%%%%%%%%%%%%%%%%%%%%
%% Note: Authors migrating a paper from PACMPL format to traditional
%% SIGPLAN proceedings format must update the '\documentclass' and
%% topmatter commands above; see 'acmart-sigplanproc-template.tex'.
%%%%%%%%%%%%%%%%%%%%%%%%%%%%%%%%%%%%%%%%%%%%%%%%%%%%%%%%%%%%%%%%%%%%%%


%% Some recommended packages.
\usepackage{booktabs}   %% For formal tables:
                        %% http://ctan.org/pkg/booktabs
\usepackage{subcaption} %% For complex figures with subfigures/subcaptions
                        %% http://ctan.org/pkg/subcaption

\usepackage[utf8]{inputenc}
\usepackage[T1]{fontenc}
\usepackage[scaled=0.8]{beramono}
\usepackage{amsmath}
\usepackage{color}
\usepackage{xcolor,colortbl}
\usepackage{url}
\usepackage{listings}
\usepackage{paralist}
%\usepackage[compact]{titlesec}
\usepackage{caption}
\usepackage{wrapfig}
\usepackage{enumitem}
\usepackage{multicol}
\usepackage{flushend}
\usepackage{textcomp}
\usepackage{pdfpages}
\usepackage{cleveref}
\usepackage{xspace}
\definecolor{light-gray}{gray}{0.85}

% ----- listings

\definecolor{light-gray}{gray}{0.85}
%\definecolor{ckeyword}{HTML}{7F0055}
\definecolor{ckeyword}{HTML}{000000}
\definecolor{ccomment}{HTML}{3F7F5F}
\definecolor{cstring}{HTML}{2A0099}

\colorlet{shadecolor}{light-gray!40}

\lstdefinelanguage{Solidity}%
{morekeywords = {
  abstract, contracts, is,
  ether, szabo, finney, wei,
  seconds, minutes, hours, days, weeks, years,
  function, constructor, memory, calldata,
  for, if, new
  mapping,
  internal, external, pure, view, payable, return, returns,
  public, private, virtual, override,
  true, false, bool,
  bytes, bytes32,
  uint, uint8, uint16, uint32, uint64, uint128, uint256,
  int, int8, int16, int32, int64, int128, int256,
  address, string,
  require, assert, revert, try, catch,
  requires, ensures, where
  },%
  sensitive,%
  %identifierstyle = \color{blue},
  moredelim = *[l][\itshape]{\#},
  morecomment = [l]//,%
  morecomment = [s]{/*}{*/},%
  morestring = [b]",%
  %morestring=[b]',%
  showstringspaces = false%
}[keywords,comments,strings]%

\lstset{
  language=Solidity,%
  backgroundcolor=\color{shadecolor},%
  mathescape=true,%
%  columns=[c]fixed,%
  aboveskip=2pt,%\smallskipamount,
  belowskip=1pt,%\negsmallskipamount,
  lineskip=-1pt,
  basewidth={0.6em, 0.5em},%
%  backgroundcolor=\color{listingbg},
  basicstyle=\ttfamily\small,
  keywordstyle=\keywordstyle,
  commentstyle=\commentstyle,
  stringstyle=\stringstyle,
%  xleftmargin=0.5cm
  literate={-->}{{$\to$}}2
           {->}{{$\mapsto$}}3
           {<-}{{$\leftarrow$}}2
           {=>}{{$\Rightarrow ~$}}2
           {=/>}{{$\not\Rightarrow ~$}}3
           {~>}{{$\leadsto$}}2
           {~/>}{{$\not\leadsto$}}2
           {|-}{{$\ts$}}2
           {σ}{{$\sigma$}}1
           {ρ}{{$\rho$}}1
           {→}{{$\to$}}1
           {←}{{$\leftarrow$}}1
           {λ}{{$\lambda$}}1
           {α}{{$\alpha$}}1
           {⊔}{{$\sqcup$}}1
           {⊓}{{$\sqcap$}}1
           {⊑}{{$\sqsubseteq$}}1
           {⊤}{{$\top$}}1
           {⊥}{{$\bot$}}1
           {×}{{$\times$}}1
           {τ}{{$\tau$}}1
           {ψ}{{$\psi$}}1
           {Σ}{{$\Sigma$}}1
           {⟨}{{$\langle$}}1
           {⟩}{{$\rangle$}}1
           {π}{{$\pi$}}1
           {∪}{{$\cup$}}2
           {/\\}{{$\land$}}2
           %{[[}{{$[\![$}}1
           %{]]}{{$]\!]$}}1
           %{…}{{$\!...$}}1
}

\definecolor{listingbg}{RGB}{240, 240, 240}

\newcommand{\commentstyle}[1]{\color{ccomment}\itshape{#1}}
\newcommand{\keywordstyle}[1]{\color{ckeyword}\bfseries{#1}}
\newcommand{\stringstyle}[1]{\color{cstring}\text{#1}}

\lstnewenvironment{listing}{\lstset{language=Solidity}}{}
\lstnewenvironment{listingtiny}{\lstset{language=Solidity,basicstyle=\scriptsize\ttfamily}}{}

\newcommand{\code}[1]{\lstinline[language=Solidity,columns=fixed,basicstyle=\ttfamily]|#1|}

\renewcommand{\paragraph}[1]{\vspace{0.15em}\noindent\textit{\textbf{#1.}}}

% ----- formal

\newcommand{\judgement}[2]{{\bf #1} \hfill #2}
\newcommand{\den}[1]{\llbracket~#1~\rrbracket}

% ----- comments and todo

\newcommand{\note}[1]{{\color{red}[#1]}}
\newcommand{\todo}[1]{\note{TODO: #1}}
\newcommand{\wac}[1]{{\color{orange}[WAC: #1]}} % Wuqi's comment
\newcommand{\zz}[1]{{\color{blue}[ZZ: #1]}} % ZZ's comment
\newcommand{\yy}[1]{{\color{Plum}[YY: #1]}} % YY's comment
\newcommand{\rev}[1]{\note{Revision: #1}}
\newcommand{\silent}[1]{}
\newcommand{\hl}[1]{\setlength{\fboxsep}{0pt}\colorbox{light-gray}{#1}}

\newcommand{\lang}{\textsc{ConSol}\xspace}
\newcommand{\corelang}{$\lambda_\lang$\xspace}

\newcommand{\lb}{\{~}
\newcommand{\rb}{~\}}
\newcommand{\la}{\langle}
\newcommand{\ra}{\rangle}

\newcommand{\Typ}[1]{\ensuremath{\mathsf{#1}}}
\newcommand{\Keywd}[1]{\ensuremath{\texttt{#1}}}
\newcommand{\Ast}[1]{\ensuremath{\mathsf{#1}}}
\newcommand{\Def}[1]{\ensuremath{\mathsf{#1}}}

% ----- syntax
\newcommand{\optional}[1]{{\color{gray}\left[{\color{black}#1}\right]}}
\newcommand{\state}{\sigma} % global state of all contracts
\newcommand{\store}[1]{\delta_{#1}} % state of a individual contract

\newcommand{\addr}{\mathcal{A}}
\newcommand{\bddr}{\mathcal{B}}
\newcommand{\cddr}{\mathcal{C}}
\newcommand{\dddr}{\mathcal{D}}
\newcommand{\xddr}{\mathcal{X}}
\newcommand{\yddr}{\mathcal{Y}}
\newcommand{\sender}{\mathtt{sender}}

% ------ semantics
\newcommand{\denot}[1]{\llbracket#1\rrbracke}


\begin{document}

%% Title information
\title[Contracts for Contracts]{Contracts for Contracts}         %% [Short Title] is optional;
%\title[Short Title]{Contracts meet Contracts}         %% [Short Title] is optional;
                                        %% when present, will be used in
                                        %% header instead of Full Title.
%\titlenote{with title note}             %% \titlenote is optional;
                                        %% can be repeated if necessary;
                                        %% contents suppressed with 'anonymous'
%\subtitle{Adding Higher-Order Contracts for Smart Contracts}                     %% \subtitle is optional
%\subtitle{Expressive Assertion Monitoring System for Safer Smart Contracts}                     %% \subtitle is optional
\subtitle{Securing Smart Contracts with Higher-Order Temporal Behavioral Contracts}                     %% \subtitle is optional
%\subtitlenote{with subtitle note}       %% \subtitlenote is optional;
                                        %% can be repeated if necessary;
                                        %% contents suppressed with 'anonymous'


%% Author information
%% Contents and number of authors suppressed with 'anonymous'.
%% Each author should be introduced by \author, followed by
%% \authornote (optional), \orcid (optional), \affiliation, and
%% \email.
%% An author may have multiple affiliations and/or emails; repeat the
%% appropriate command.
%% Many elements are not rendered, but should be provided for metadata
%% extraction tools.

%% Author with single affiliation.
\author{First1 Last1}
\authornote{with author1 note}          %% \authornote is optional;
                                        %% can be repeated if necessary
\orcid{nnnn-nnnn-nnnn-nnnn}             %% \orcid is optional
\affiliation{
  \position{Position1}
  \department{Department1}              %% \department is recommended
  \institution{Institution1}            %% \institution is required
  \streetaddress{Street1 Address1}
  \city{City1}
  \state{State1}
  \postcode{Post-Code1}
  \country{Country1}                    %% \country is recommended
}
\email{first1.last1@inst1.edu}          %% \email is recommended

%% Author with two affiliations and emails.
\author{First2 Last2}
\authornote{with author2 note}          %% \authornote is optional;
                                        %% can be repeated if necessary
\orcid{nnnn-nnnn-nnnn-nnnn}             %% \orcid is optional
\affiliation{
  \position{Position2a}
  \department{Department2a}             %% \department is recommended
  \institution{Institution2a}           %% \institution is required
  \streetaddress{Street2a Address2a}
  \city{City2a}
  \state{State2a}
  \postcode{Post-Code2a}
  \country{Country2a}                   %% \country is recommended
}
\email{first2.last2@inst2a.com}         %% \email is recommended
\affiliation{
  \position{Position2b}
  \department{Department2b}             %% \department is recommended
  \institution{Institution2b}           %% \institution is required
  \streetaddress{Street3b Address2b}
  \city{City2b}
  \state{State2b}
  \postcode{Post-Code2b}
  \country{Country2b}                   %% \country is recommended
}
\email{first2.last2@inst2b.org}         %% \email is recommended

\lstMakeShortInline[keywordstyle=,%
                    flexiblecolumns=false,%
                    %basewidth={0.56em, 0.52em},%
                    mathescape=false,%
                    basicstyle=\ttfamily]@

%% Abstract
%% Note: \begin{abstract}...\end{abstract} environment must come
%% before \maketitle command
\begin{abstract}
Text of abstract \ldots.
\end{abstract}


%% 2012 ACM Computing Classification System (CSS) concepts
%% Generate at 'http://dl.acm.org/ccs/ccs.cfm'.
\begin{CCSXML}
<ccs2012>
<concept>
<concept_id>10011007.10011006.10011008</concept_id>
<concept_desc>Software and its engineering~General programming languages</concept_desc>
<concept_significance>500</concept_significance>
</concept>
<concept>
<concept_id>10003456.10003457.10003521.10003525</concept_id>
<concept_desc>Social and professional topics~History of programming languages</concept_desc>
<concept_significance>300</concept_significance>
</concept>
</ccs2012>
\end{CCSXML}

\ccsdesc[500]{Software and its engineering~General programming languages}
\ccsdesc[300]{Social and professional topics~History of programming languages}
%% End of generated code


%% Keywords
%% comma separated list
\keywords{keyword1, keyword2, keyword3}  %% \keywords are mandatory in final camera-ready submission


%% \maketitle
%% Note: \maketitle command must come after title commands, author
%% commands, abstract environment, Computing Classification System
%% environment and commands, and keywords command.
\maketitle

%https://se.inf.ethz.ch/~meyer/publications/old/dbc_chapter.pdf
%https://docs.soliditylang.org/en/v0.8.17/control-structures.html?highlight=require#panic-via-assert-and-error-via-require

\section{Introduction}

%\todo{why? solidity is rather flexible, unsatisfying/weaker type system,
%not all invariants can be checked statically, e.g. the status of the chain}
%One \note{common} use case of higher-order functions in smart contracts is
%to express callback \todo{explain how it is used}.

%\todo{it is hard to enforce properties when callbacks exist}
%Static verification cannot completely address the issue here since
%transaction developers are oblivious to the user-provided callback.

%Building trustworthy and secure smart contract programs requires solid
%understanding and reliable enforcement of program invariants.

Smart contracts are programs to automate the execution of agreement so that all
parties involved in the agreement can be certain about the outcome.
Smart contracts are ``smart'' since by using blockchains they provide a
mechanism to \emph{ensure the economic outcome of transactions} without a
central authority.
A blockchain is a \note{XXX} that implements a consensus protocol of
Byzantine-fault-tolerant distributed ledgers.

It is however ironic that many popular smart contract languages, which hold the
great promise to ensure transactional outcomes via consensus and distributed
ledgers, do not provide enough means to \emph{ensure the computational outcomes},
leading to poor quality of programs.
Without ensuring the quality of programs, it is impossible to ensure
the economic outcome of transactions.

Indeed, we have seen a great number of notorious attacks to smart contract
systems in recent years.
For example, \todo{explain attack using callback}.
Attackers are able to utilize
the callback mechanism to \note{do crazy things and steal money}
The root cause of this attack is \todo{explain}.

Solidity does provide first-order assertion primitives (e.g. @require@ or
@modifier@) but they cannot effectively check function values, which are a
common pattern used in smart contract programs. Programmers have to
manually insert such checks around relevant call-sites, \todo{which is bad}.
%Solidity also provides the @modifier@ mechanism that intercepts
%before and after a function call, thus can be used to check
%pre- and post-conditions.
%\todo{can we compose modifier?}
%\todo{can we attach modifier to callbacks?}
%\todo{can we use modifier to check function's argument?}

The lack of sufficient linguistic means to specify and enforce subtle behaviors
discourages programmers to use concise higher-level abstractions when writing
smart contract code.
Instead, to be secure, programmers are obliged to write verbose code with
lower-level abstractions, which conflates the main business logic and
behavioral checks, leading to poor readability and maintainability.
For instance, to defend the famous DAO attack \note{cite} caused by allowing
``reentrancy'' via external functions, programmers have to be extra cautious
about the execution order or maintain additional states to check if a call is
allowed. Either way is not ideal.
And often worse, defensive checks can be easily forgotten or disregarded,
leading to vulnerable code causing real monetary lose.

Apart from investigating effort to audit, certify, or verify smart contract programs,
language designers are responsible to empower programmers with better
linguistic solutions to secure smart contracts in the first place.
In this paper, we show that advanced \emph{behavioral software contracts} provide a
lightweight solution for securing \emph{smart contracts}.

\subsubsection*{\textbf{Behavioral Contracts for Smart Contracts}}
Behavioral contracts provide a mechanism for programmers to specify assumptions
and guarantees of software components. The specified assumptions and guarantees
will be monitored and reported when violated.
Pioneered by the Eiffel language \cite{DBLP:books/ph/Meyer91} and the
Design-by-Contract methodology \cite{DBLP:conf/tools/Meyer98a}, behavioral
contracts have been widely adopted in Java, Haskell, C\#, Python, Racket,
Elixir \cite{DBLP:conf/erlang/0001BBHMEF22}, etc. \todo{cite}

\todo{why BC is good here?}

At the conceptual level, behavioral contracts is complementary to smart
contracts, in the sense that they are both agreement -- between software
components for the former, and between transactional parties for the later.
Thus, the two terms ``contract'' indeed mean the same thing but just different
levels of enforcement.

In this work, we design and implement a practical extension of Solidity, the
most popular language for building smart contract programs. The extension
enriches the Solidity language with \emph{higher-order} and \emph{temporal}
behavioral contracts, which are absent in existing approaches.
\todo{allows furthermore downstream tasks, fuzzing, static verification,
natural language to formal spec etc.}

\subsubsection*{\textbf{Contracts for Higher-Order Functions}}
Higher-order contracts provides a way to specify and enforce pre- and
post-conditions of untrusted functions. \todo{...}

\todo{restrict possible call targets}
\todo{why static verification fails}

\subsubsection*{\textbf{Contracts for Temporal Properties}}
Temporal contracts provides a way to \todo{...}.
\todo{why static verification fails}


We develop a core model of the extension, including its syntax,
operational semantics, and discuss its soundness guarantee.
With this extension of higher-order temporal behavioral contracts, we examine
several cases of reported bugs and vulnerabilities and show that
our approach is effective in preventing these attacks.
We also evaluate the overhead (gas consumption) of our approach
compared to manually inserted assertion checks, showing that \todo{...}

%\todo{discuss performance/gas consumption}

%\todo{allows interesting programming idioms to be expressed}

% second-class contracts

\subsubsection*{\textbf{Contributions}} This paper makes the following contributions:
\begin{itemize}
  \item We propose a specification of behavioral contracts for expressing and
    enforcing higher-order and temporal properties for smart contracts.
  \item We design \lang, an extension for Solidity with
    behavioral contracts, demonstrate its use cases through extensive examples,
    and how it can improve the safety and reliability.
  \item We present the core formalization $\lambda_\lang$ for our proposed
    extension, including its syntax, formal semantics, compilation, and
    soundness.
  \item We examine a number of real-world attacks and show that our proposal is
    effective in improving safety and reliability.
  \item We implement our approach on top of the existing Solidity compiler and shows
    that its runtime overhead is marginal, therefore practical to use.
\end{itemize}

\section{Contract Solidity by Examples}

In this section, we introduce \lang informally with examples.
\lang is a non-breaking extension that additionally provides means
to specify behavioral contracts on top of the Solidity language. We provide
a brief introduction of Solidity first, then discuss our extension of behavioral
contracts.

\subsection{The Solidity Language}

brief introduction of the Solidity language, why it is complicated

\subsubsection*{\textbf{\lang's Design Rationale}}
This paper concerns about designing an extension for Solidity that enhances the
language with higher-order temporal behavioral contracts.
The extension should be
\todo{non-intrusive, easy to use, expressive, gradual (you do not need to pay the overhead
if you don't use contracts), efficient (little overhead)}

\subsection{Flat Contracts for First-Order Values}

Specifying the pre-condition and post-condition, where the post-condition
can depend on the argument value:

\begin{lstlisting}[language=Solidity]
  # {x | x > 0}
  # {y | y > x + 1}
  function f(uint x) returns y { ... }
\end{lstlisting}
It is also equivalent to core syntax:
\begin{lstlisting}
  # {x | x > 0} -> {y | y > x + 1}
\end{lstlisting}

We can omit any part of the argument or return value contracts:

\begin{lstlisting}[language=Solidity]
  # {x | x > 0}
  function f(uint x) returns y { ... }
\end{lstlisting}
It is equivalent to core syntax where the omitted part is simply \code{true}:

\begin{lstlisting}
  # {x | x > 0} -> {y | true}
\end{lstlisting}
Another way is to use the `any` flat contract, which is defined as
\code{any = { _ | true }}:
\begin{lstlisting}
  # {x | x > 0} -> any
\end{lstlisting}

For multi-argument functions, we use multiple \code{->} to compose larger function
contracts:
\begin{lstlisting}
  # {x | x > 0} -> {y | y < 100} -> {z | z == x + y }
  function f(int x, int y) returns (int) { returns x + y; }
\end{lstlisting}
Or, we could write a single flat contract for the argument introducing
multiple argument binders:
\begin{lstlisting}
  # { (x, y) | x > 0 && y < 100 } -> {z | z == x + y }
  function f(int x, int y) returns int { returns x + y; }
\end{lstlisting}

When specifying the spec of functions, two special variables are available
`msg.sender` the address of the caller and `block.timestamp` the current
block timestamp.

\begin{lstlisting}
  # { msg.sender ... }
  NEED AN EXAMPLE USING THEM
\end{lstlisting}

Behavioral contracts for first-order values are straightforward,
and can be directly mapped to \code{assert} or \code{require} statements
appearing at the beginning of the function.

\subsection{Contracts for Higher-Order Functions}

\iffalse
\begin{lstlisting}
function map(uint[] memory data, function (uint) pure returns (uint) f)
  internal pure returns (uint[] memory r)
{
  r = new uint[](data.length);
  for (uint i = 0; i < data.length; i++) {
    r[i] = f(data[i]);
  }
}
\end{lstlisting}
\fi

We can specify the contract for function arguments too:
\begin{lstlisting}
  # { f | {x | x < 0} -> {y | y > 0} }
  function map(int[] memory data, function (int) pure returns (int) f) { ... }
\end{lstlisting}
It might be too verbose -- so we can define those predicates separately for better readability/maintainability:
\begin{lstlisting}
  function greaterThanZero(int x) returns (bool) {
    return x > 0;
  }
  function smallerThanZero(int x) returns (bool) {
    return x < 0;
  }
  # { f | smallerThanZero -> greaterThanZero }
  function map(int[] memory data, function (int) pure returns (int) f) { ... }
\end{lstlisting}

Functions contracts can be higher-order -- it can take other function contracts
as part of the spec. For example
\begin{lstlisting}
  # TODO
  function f(function (function (int) returns (int) g) h) { ... }
\end{lstlisting}

Function contracts are first-class -- so if this guarded function is escaped
(e.g. by returning), the contract is still enforced:
\begin{lstlisting}
NEED AN EXAMPLE
\end{lstlisting}


\subsection{Contracts for Addresses}

Address contracts share similar syntax as function contracts, but introduces
additionally more binders for the arguments and returned values:

\begin{lstlisting}
  # { a | { {value, gas, ...} | <pre-cond-value-gas> }
      -> { arg | <pre-cond-arg> }
      -> { (res, data) | <post-cond> } }
  function f(address a) {
    (bool success, bytes memory data) = a.call{value: ...}(arg);
  }
\end{lstlisting}

It would be convenient to directly enforces the success of the call and omit
the data:
\begin{lstlisting}
  # { a | { {value, gas}(arg) | <pre-cond> } -> { (true, _) | true } }
  function f(address a) { ... }
\end{lstlisting}

Address call can designate the callee function, therefore it is necessary to
specify the spec for possible callees.

\begin{lstlisting}
  # { a.f | pre-cond -> post-cond }
  # { a.g | pre-cond -> post-cond }
  function f(address a) { ... }
\end{lstlisting}

\subsubsection*{Limitations}
Address contracts are \emph{not} higher-order -- if an address takes another
address as argument, then we cannot enforce the contract of the argument
address.
For example, in the following case, @b@ itself has an ``address contract''
@<b-pre> -> <b-post>@ but there is no general way to enforce that.
Because we do not have the control of the actual callee function of @a@,
therefore no way to modify the code of @a@ to check the pre-condition
and post-condition when @b@ is called in @a@.

\begin{lstlisting}
  # { a | { b | <b-pre> -> <b-post> } -> { (true, _) | true } }
  function f(address a) {
    address b = ...
    (bool success, bytes memory data) = a.call(b);
  }
\end{lstlisting}

Address contracts are \emph{second-class} -- we can only enforce the
pre-condition and post-condition of addresses within the current calling
context.  If the address escapes (e.g. by returning, or stored in a global
variable), we cannot enforce the contract anymore.

Why? Because in general we do not have the ability to insert checks around the
computation of that address. We can indeed create a new contract (and a new address) that wraps the old address call with checks, but then calling
the new address will not exhibit the same behavior as calling the old address
for the caller.

For example, the following identity function of addresses has a spec
for the argument address. But the body of function does not invoke
the address, instead, it directly returns @a@ to the caller.
Once address @a@ escapes, we cannot enforce the condition anymore.

\begin{lstlisting}
  # { a | { {value, gas}(arg) | <pre-cond> } -> { (res, data) | <post-cond> } }
  function f(address a) returns address { return a }
\end{lstlisting}

\subsection{Temporal Properties}

Temporal contracts relate two function calls (events) and enforces their
relation.
They can inspect the time-stamps, message sender, arguments, return values
of the two function calls.

\todo{where to add message sender? proposed syntax:}
\begin{lstlisting}
  # ts1 @ f(x) returns z from sdr1 => ts2 @ g(a) returns c from sdr2
\end{lstlisting}

\subsubsection{Positive Temporal Properties}

Positive temporal properties enforces that an event must happen
after another event. Syntax:
\begin{lstlisting}
  # ts1 @ f(x) returns z => ts2 @ g(a) returns c /\ side-cond
\end{lstlisting}
It enforces that when @g@ is invoked, we check @f@ has invoked \emph{and}
the side-condition is evaluated to true.
We do not enforce that @g@ will be invoked eventually after invoking @f@.

Side conditions can be arbitrary expressions that uses the time-stamps, message
sender, arguments, return values of the two function calls.
For example, the following spec passes arguments of @f@ and @g@ into
another function for checking:

\begin{lstlisting}
  # ts1 @ f(x) returns z => ts2 @ g(a) returns c /\ check(x, a)
\end{lstlisting}

If a variable occurs in both events, an additional equality constraint
is synthesized.  For example,
\begin{lstlisting}
  # ts1 @ f(x) returns z => ts1 @ g(z) returns c
\end{lstlisting}
requires that both function @f@ and @g@ happen in the same transaction,
and the return value of @f@ is the same as the argument of @g@.
It will be translated to the core syntax as the following:
\begin{lstlisting}
  # ts1 @ f(x) returns z => ts2 @ g(y) returns c /\ ts1 == ts2 && x == y
\end{lstlisting}

It is totally fine to omit the time-stamp, arguments, or return values.
Simply requiring @f@ happens before @g@:
\begin{lstlisting}
  # f => g
  function f(...) { ... }
\end{lstlisting}

\todo{need to think about quantification.}

Note: @f => g@ matches the most recent call of @f@,
or any prior call of @f@ (if there are multiple calls of @f@)?

\todo{need to think about interactions with compiler optimization (eg inlining).}

\todo{need to think about how temporal contracts interact with address calls.}

\subsubsection{Negative Temporal Properties}

Negative temporal properties enforces that an event must \emph{not}
happen after another event.

\begin{lstlisting}
  # ts1 @ f(x) returns z =/> ts2 @ g(a) returns c /\ side-cond
\end{lstlisting}
It enforces that when @g@ is invoked, we check that @f@ has not invoked
\emph{and} the side condition is evaluated to true.
We do not enforce that @g@ cannot be invoked eventually
after invoking @f@.

For example, the following spec enforces that @g@ cannot be
invoked after @f@ using @z@ as the argument:
\begin{lstlisting}
  # ts1 @ f(x) returns z =/> ts2 @ g(z) returns c
  //or equivalently
  # ts1 @ f(x) returns z =/> ts2 @ g(y) returns c /\ x == y
\end{lstlisting}
However it is okay to call @g@ after @f@ with an argument different from @z@.

\subsection{Contextual Properties}

Contextual contracts inspect the calling-context/stack within the
current smart contract.

\note{Related work: stack inspection. But we don't need to modify the runtime
of EVM, but simply record a shadow stack for necessary metadata. Is that enough?}

\subsubsection{Positive Contextual Properties}

For example, the following spec enforces that @g@ can only be invoked
under the calling context of @h@, and the return value of @h@ must be
the same as the argument of @g@:

\begin{lstlisting}
  function h(int x) returns int { ... }
  # h(x) returns y ~> g(y) returns r
  function g(int y) returns int { ... }
\end{lstlisting}

To enforce this behavioral contracts, we must check
\begin{itemize}
  \item when @g@ is invoked, there is a frame of @h@ on the stack,
  \item when @h@ returns (which happens after @g@ returns), the return value
  is the same as the argument of @g@.
\end{itemize}

Question: what if @g@ is invoked multiple times under the same calling context of
@h@?

Solution: need quantifier. A simple/default solution is to check all
invocations of @g@.

Question: what if there are multiple call frames of @h@ on the stack?

Solution: need quantifier. A simple/default solution is to only check
the most recent call frame of @h@ on the stack.

\subsubsection{Negative Contextual Properties}

The following program is a violation of non-reentrancy:

\begin{lstlisting}
  # non-reentrant
  function f(uint x) { f(n) }
\end{lstlisting}

This is an example of negative contextual properties, i.e. something cannot
happen under the current calling context. Under the neath, this spec is
equivalent to

\begin{lstlisting}
  # f ~/> f
  function f(uint x) { f(n) }
\end{lstlisting}

\subsection{Substructural Properties}

affine (at most call once)
%linear (must call once),
%relevant (at least call once)

\section{Formal Model}

the core model of the language $\lambda_\lang$
\todo{there must be some interesting points/trade-offs about the design}

\subsection{Syntax}

\newcommand{\AddrCall}[3]{
  {#1}\Keywd{\{}{#2}\Keywd{\}(}{#3}\Keywd{)}
}
\newcommand{\Call}[2]{
  {#1}\Keywd{(}{#2}\Keywd{)}
}
\newcommand{\SpecCall}[4]{
  {#1}\Keywd{\{}{#2}\Keywd{\}(}{#3}\Keywd{)} ~\Keywd{returns (}{#4}\Keywd{)}
}
\newcommand{\SpecCond}[3]{
  \Keywd{requires}~ {#1} ~\Keywd{ensures}~ {#2} ~\Keywd{where}~ {#3}
}
\newcommand{\FunDef}[6]{
  {#1} ~\Keywd{function}~ {#2}\Keywd{(}{#3}\Keywd{)} ~ {#4} ~\Keywd{returns (}{#5}\Keywd{)} ~\Keywd{\{}~ {#6} ~\Keywd{\}}
}
\newcommand{\FunType}[4]{
  \Keywd{function}~ {#1}\Keywd{(}{#2}\Keywd{)} ~ {#3} ~\Keywd{returns (}{#4}\Keywd{)}
}
\newcommand{\Contract}[3]{
  \Keywd{contract}~ {#1}~ \Keywd{\{}~ {#2}; {#3} ~\Keywd{\}}
}
\newcommand{\Interface}[2]{
  \Keywd{interface}~ {#1}~ \Keywd{\{}~ {#2} ~\Keywd{\}}
}

\newcommand{\F}{\mathcal{F}}
\newcommand{\C}{\mathcal{C}}
\newcommand{\I}{\mathcal{I}}

\begin{figure}
  \begin{alignat*}{3}
    &~ n && \in && \mathbb{Z} \qquad b \in \mathbb{B} \qquad x,y,f \in \Typ{Id}   \\
    \Typ{Data Types} &~ t && :=\ && \Keywd{int} \mid \Keywd{uint} \mid \Keywd{bool} \mid \Keywd{address} \\
                &~   && \mid\ && \Keywd{mapping} ~t~ \Keywd{=>} ~t \mid \Keywd{struct} ~x~ \Keywd{\{}~ d^* ~\Keywd{\}} \\
    \Typ{Type Decl}    &~ d && :=\ && t ~ x \\
    \Typ{Values}      &~ v && :=\ && n \mid b \\
    \Typ{Projection}  &~ p && :=\ && \Keywd{.}x \mid \Keywd{[}e\Keywd{]} \\
    \Typ{Expressions} &~ e && :=\ && x \mid v \mid e ~op~ e \mid e p^+ \mid \AddrCall{e}{(x: e)^*}{e^*} \\
    \Typ{Assignable}  &~ a && :=\ && d \mid x \mid e p^+ \\
    \Typ{Statements}  &~ s && :=\ && d \mid e \mid s; s \mid a ~\Keywd{=}~ e \mid \Keywd{return}~ e \mid \Keywd{revert} \\
                      &~   && \mid\ && \Keywd{if (}e\Keywd{) \{}~ s ~\Keywd{\} else \{}~ s ~\Keywd{\}} \\
    \Typ{Spec}        &~ \sigma && :=\ && \SpecCall{fp^*}{(x : y)^*}{x^*}{y^*} \\
                      &~        &&     && \SpecCond{e}{e}{\sigma^*} \\
    \Typ{Modifiers}   &~ m && :=\ && \Keywd{public} \mid \Keywd{private} \\
    \Typ{Fun Decl}    &~ d_f && :=\ && \FunType{f}{d^*}{m}{t^*} \\
    \Typ{Fun Def}     &~ \F&& :=\ && \sigma ~ d_f ~ \Keywd{\{} ~ s \Keywd{\}} \\
    \Typ{Contract}    &~ \C&& :=\ && \Contract{x}{d^*}{\F^*} \\
    \Typ{Interface}   &~ \I&& :=\ && \Interface{x}{d_f^*}
  \end{alignat*}
  % TODO(GW): spec callee with interface
  \caption{The abstract syntax of $\lambda_\lang$.}
  \label{fig:syntax}
\end{figure}

\iffalse

  \begin{alignat*}{3}
    \Typ{Types}      \qquad & t && ::=
      \Keywd{address} \mid
      \Keywd{uint} \mid
      \Keywd{function}(t, \dots)~\Keywd{returns}~t \mid
      \dots \\
    \Typ{Literals}    \qquad & v && \in
      \mathbb{Z} \mid
      \todo{whatelse} \\
    \Typ{Operators} \qquad & o && ::=
      + \mid
      - \mid
      \times \mid
      \land \mid
      \lor \mid
      \dots \\
    \Typ{Expressions}\qquad & e && ::=
      v \mid
      x \mid
      o(e, \dots) \mid
      \todo{} \\
    \Typ{Statements} \qquad & s && ::=
                                   \Keywd{revert} \mid
                                   s_1; s_2 \mid
                                   x ~\mathsf{:=}~ e \mid
                                   x ~\mathsf{:=}~ f(e, \dots)  \text{\yy{unify assignments?}}\\
                            & && \mid \Keywd{if}~e~\Keywd{then}~s_1~\Keywd{else}~s_2 \mid
                                   \Keywd{while}~e~\Keywd{do}~s
                                   \\
    \Typ{Temporal~Connectives} \qquad & \oplus && ::=
      ~ \Rightarrow ~\mid~ \not\Rightarrow ~\mid~ \leadsto ~\mid~ \not\leadsto \\
    \Typ{Value~Specifications} \qquad & \varsigma_v && ::=
      \{ (x, \dots) ~\vert~ e \} \mid
      \varsigma_v \mapsto \varsigma_v \\
    \Typ{Temporal~Specifications} \qquad & \varsigma_t && ::=
     f(x, \dots)~\Keywd{returns}~ (z_1, \dots) ~\oplus~ g(y, \dots)~\Keywd{returns}~ (z_2, \dots) \land e \\
    \Typ{Specifications} \qquad & \varsigma && ::= \varsigma_v \mid \varsigma_t \\
    \Typ{Function~Definitions} \qquad & \mathcal{F} && ::=
      \# \varsigma^* ~ \Keywd{function}~f(t ~ x, \dots)~\Keywd{returns}~t~\{ ~s~ \}
  \end{alignat*}

\fi
%%% Local Variables:
%%% mode: latex
%%% TeX-master: "paper"
%%% End:


\subsection{Semantics}

static semantics

dynamic semantics

\subsection{Compilation of Specification}

\subsection{Soundness}

what does it guarantee

\section{Case Studies}

examining 3-4 cases, they are the proof that the approach works

\subsection{Reentrancy}

\subsection{Callback}

\subsection{Comparison with \texttt{modifier}}

\todo{what else?}

\section{Implementation}

the implementation that modifies the off-the-shelf Solidity compiler that
supports contracts and generates code with correct contracts checking

\todo{what we didn't handle: inlined assembly, and ??? }

\section{Performance Evaluation}

\section{Related Work}

\subsubsection*{\textbf{Behavioral Contracts}}

Eiffel the ``design-by-contract'' methodology \cite{DBLP:books/ph/Meyer91, DBLP:conf/tools/Meyer98a}

higher-order contract \cite{DBLP:conf/icfp/FindlerF02}

temporal higher-order contract \cite{DBLP:conf/icfp/DisneyFM11}

\subsubsection*{\textbf{Smart Contracts}}

core calculus of Solidity-like languages \cite{Sergey2021, DBLP:conf/esorics/BartolettiGM19, DBLP:conf/fc/CrafaPZ19}

runtime validation, reduce overhead of runtime checks \cite{DBLP:conf/pldi/LiCL20}

security analyzer \cite{DBLP:conf/pldi/BrentGLSS20}

nondeterminisitic payment bugs \cite{DBLP:journals/pacmpl/WangZS19}

behavioral simulation to verify smart contracts \cite{DBLP:conf/pldi/BeillahiCEE20}

formal and modular specification of smart contracts \cite{DBLP:journals/pacmpl/BramEMSS21}

static analysis to infer ownership and commutativity summaries, for parallelism
\cite{DBLP:conf/pldi/Pirlea0S21}

callbacks \cite{DBLP:journals/pacmpl/AlbertGRRRS20, DBLP:journals/pacmpl/GrossmanAGMRSZ18}

reduce gas consumption \cite{DBLP:journals/pacmpl/GrechKJBSS18}

static analysis of Ethereum \cite{DBLP:journals/pacmpl/SmaragdakisGLTT21}

\section{Conclusion}


%% Acknowledgments
\begin{acks}                            %% acks environment is optional
                                        %% contents suppressed with 'anonymous'
  %% Commands \grantsponsor{<sponsorID>}{<name>}{<url>} and
  %% \grantnum[<url>]{<sponsorID>}{<number>} should be used to
  %% acknowledge financial support and will be used by metadata
  %% extraction tools.
  This material is based upon work supported by the
  \grantsponsor{GS100000001}{National Science
    Foundation}{http://dx.doi.org/10.13039/100000001} under Grant
  No.~\grantnum{GS100000001}{nnnnnnn} and Grant
  No.~\grantnum{GS100000001}{mmmmmmm}.  Any opinions, findings, and
  conclusions or recommendations expressed in this material are those
  of the author and do not necessarily reflect the views of the
  National Science Foundation.
\end{acks}


%% Bibliography
\bibliography{references}


%% Appendix
%\appendix
%\section{Appendix}

Text of appendix \ldots

\end{document}
