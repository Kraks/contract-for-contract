
\section{Contract Solidity by Examples}

In this section, we introduce \lang informally with examples.
\lang is a non-breaking extension that additionally provides means
to specify behavioral contracts on top of the Solidity language.
We provide a brief introduction of Solidity first, then discuss our extension
of behavioral contracts.
We often use \emph{specification} for behavioral contracts to avoid terminological ambiguity.

\subsection{The Solidity Language}

brief introduction of the Solidity language, why it is complicated

\subsubsection*{\textbf{\lang's Design Rationale}}
This paper concerns about designing an extension for Solidity that enhances the
language with higher-order temporal behavioral contracts.
The extension should be
\todo{non-intrusive, easy to use, expressive, gradual (you do not need to pay the overhead
	if you don't use contracts), efficient (little overhead)}

\subsection{Flat Contracts for First-Order Values}

Specifying the pre-condition and post-condition, where the post-condition
can depend on the argument value:

% \begin{lstlisting}[language=Solidity]
%   # {x | x > 0}
%   # {y | y > x + 1}
%   function f(uint x) returns (uint y) { ... }
% \end{lstlisting}

\begin{lstlisting}[language=Solidity]
  # {x | x % 1e18 < 1e17}
  # {y | y < type(uint).max / 1e18}
  function toEther(uint x) returns (uint y) { ... }
\end{lstlisting}

It is also equivalent to core syntax:
% \begin{lstlisting}
%   # {x | x > 0} -> {y | y > x + 1}
% \end{lstlisting}

\begin{lstlisting}[language=Solidity]
  # {x | x % 1e18 < 1e17} -> {y | y < type(uint).max / 1e18}
  function toEther(uint x) returns (uint y) { ... }
\end{lstlisting}

We can omit any part of the argument or return value contracts:

% \begin{lstlisting}[language=Solidity]
%   # {x | x > 0}
%   function f(uint x) returns (uint y) { ... }
% \end{lstlisting}

\begin{lstlisting}[language=Solidity]
  # {x | x < type(uint).max / 1e18}
  function toWei(uint x) returns (uint y) { ... }
\end{lstlisting}

It is equivalent to core syntax where the omitted part is simply \code{true}:

\begin{lstlisting}[language=Solidity]
  # {x | x < type(uint).max / 1e18} -> {y | true}
\end{lstlisting}

Another way is to use the `any` flat contract, which is defined as
\code{any = { _ | true }}:
\begin{lstlisting}
  # {x | x < type(uint).max / 1e18} -> any
\end{lstlisting}

For multi-argument functions, we use multiple \code{->} to compose larger function
contracts:

\begin{lstlisting}
  # {x | x > 0} -> {y | y < 100} -> {z | z == x + y }
  function f(int x, int y) returns (int z) { returns x + y; }
\end{lstlisting}

% \begin{lstlisting}[language=Solidity]
% # {rate | rate > 0 && rate < 100}
% # {time | time != 0}
% function simpleInterest(uint principle, uint rate, uint time) returns (uint interest) { ... }
% \end{lstlisting}

Or, we could write a single flat contract for the argument introducing
multiple argument binders:

\begin{lstlisting}
  # { (x, y) | x > 0 && y < 100 } -> {z | z == x + y }
  function f(int x, int y) returns (int z) { returns x + y; }
\end{lstlisting}

When specifying the spec of functions, two special variables are available
`msg.sender` the address of the caller and `block.timestamp` the current
block timestamp.
\zz{How about other special variables, such as `block.chainid` and `tx.origin`?. Perhaps we could redesign a universal syntax for these variables in Section 2.5, where we discuss where to add the message sender}
\url{https://docs.soliditylang.org/en/develop/units-and-global-variables.html#block-and-transaction-properties}

\begin{lstlisting}
  # { n | msg.value >= 1e15 * n }
  function buyTickets(int n) payable { ... }
\end{lstlisting}

Behavioral contracts for first-order values are straightforward,
and can be directly mapped to \code{assert} or \code{require} statements
appearing at the beginning of the function.

\subsection{Contracts for Higher-Order Functions}

\iffalse
	\begin{lstlisting}
function map(uint[] memory data, function (uint) pure returns (uint) f)
  internal pure returns (uint[] memory r)
{
  r = new uint[](data.length);
  for (uint i = 0; i < data.length; i++) {
    r[i] = f(data[i]);
  }
}
\end{lstlisting}
\fi

We can specify the contract for function arguments too:
\begin{lstlisting}
  # { f | {x | x < 0} -> {y | y > 0} }
  function map(int[] memory data, function (int) pure returns (int) f) { ... }
\end{lstlisting}
It might be too verbose -- so we can define those predicates separately for better readability/maintainability:
\begin{lstlisting}
  function greaterThanZero(int x) pure returns (bool) {
    return x > 0;
  }
  function smallerThanZero(int x) pure returns (bool) {
    return x < 0;
  }
  # { f | smallerThanZero -> greaterThanZero }
  function map(int[] memory data, function (int) pure returns (int) f) { ... }
\end{lstlisting}

Functions contracts can be higher-order -- it can take other function contracts
as part of the spec. For example
\zz{coooool!}
\begin{lstlisting}
  # TODO
  function f(function (function (int) returns (int) g) h) { ... }
\end{lstlisting}

Function contracts are first-class -- so if this guarded function is escaped
(e.g. by returning), the contract is still enforced:
\begin{lstlisting}
NEED AN EXAMPLE
\end{lstlisting}


\subsection{Contracts for Addresses}

Address contracts share similar syntax as function contracts, but introduces
additionally more binders for the arguments and returned values:

\begin{lstlisting}
  # { a | { {value, gas, ...} | <pre-cond-value-gas> }
          -> { arg | <pre-cond-arg> }
          -> { (res, data) | <post-cond> } }
  function f(address a) {
    (bool success, bytes memory data) = a.call{value: ...}(arg);
  }
\end{lstlisting}

It would be convenient to directly enforces the success of the call and omit
the data:
\begin{lstlisting}
  # { a | { {value, gas}(arg) | <pre-cond> } -> { (true, _) | true } }
  function f(address a) { ... }
\end{lstlisting}

Address call can designate the callee function, therefore it is necessary to
specify the spec for possible callees.
\zz{just FYI, `a.f` can also specify `value`, `gas`, etc.}

\begin{lstlisting}
  # { a.f | pre-cond -> post-cond }
  # { a.g | pre-cond -> post-cond }
  function f(address a) { ... }
\end{lstlisting}

\wac{We may need an example here since the necessity to designate the callee function may not be obvious to reviewers.}
\zz{I guess we should educate the audience about the difference between low-level and high-level calls at the beginning of Section 2.4. I'll take care of this later.}

\subsubsection*{Limitations}
Address contracts are \emph{not} higher-order -- if an address takes another
address as argument, then we cannot enforce the contract of the argument
address.
For example, in the following case, @b@ itself has an ``address contract''
@<b-pre> -> <b-post>@ but there is no general way to enforce that.
Because we do not have the control of the actual callee function of @a@,
therefore no way to modify the code of @a@ to check the pre-condition
and post-condition when @b@ is called in @a@.

\begin{lstlisting}
  # { a | { b | <b-pre> -> <b-post> } -> { (true, _) | true } }
  function f(address a) {
    address b = ...
    (bool success, bytes memory data) = a.call(b);
  }
\end{lstlisting}

Address contracts are \emph{second-class} -- we can only enforce the
pre-condition and post-condition of addresses within the current calling
context.  If the address escapes (e.g. by returning, or stored in a global
variable), we cannot enforce the contract anymore.

Why? Because in general we do not have the ability to insert checks around the
computation of that address. We can indeed create a new contract (and a new address) that wraps the old address call with checks, but then calling
the new address will not exhibit the same behavior as calling the old address
for the caller.

For example, the following identity function of addresses has a spec
for the argument address. But the body of function does not invoke
the address, instead, it directly returns @a@ to the caller.
Once address @a@ escapes, we cannot enforce the condition anymore.

\begin{lstlisting}
  # { a | { {value, gas}(arg) | <pre-cond> } -> { (res, data) | <post-cond> } }
  function f(address a) returns (address) { return a }
\end{lstlisting}

\subsection{Temporal Properties}

Temporal contracts relate two function calls (events) and enforces their
relation.
They can inspect the time-stamps, message sender, arguments, return values
of the two function calls.

\todo{where to add message sender? proposed syntax:}
\begin{lstlisting}
  # ts1 @ f(x) returns z from sdr1 => ts2 @ g(a) returns c from sdr2
\end{lstlisting}
\begin{lstlisting}
  # f(x) returns z => g(a) returns c /\ f.msg.sender == g.msg.sender
\end{lstlisting}
\wac{\texttt{msg.sender} is better tied with transaction. We can use ts1.msg.sender and I prefer to put the constraints in the side-cond. }

\subsubsection{Positive Temporal Properties}

Positive temporal properties enforces that an event must happen
after another event. Syntax:
\begin{lstlisting}
  # ts1 @ f(x) returns z => ts2 @ g(a) returns c /\ side-cond
\end{lstlisting}
It enforces that when @g@ is invoked, we check @f@ has invoked \emph{and}
the side-condition is evaluated to true.
We do not enforce that @g@ will be invoked eventually after invoking @f@.

Side conditions can be arbitrary expressions that uses the time-stamps, message
sender, arguments, return values of the two function calls.
For example, the following spec passes arguments of @f@ and @g@ into
another function for checking:

\begin{lstlisting}
  # ts1 @ f(x) returns z => ts2 @ g(a) returns c /\ check(x, a)
\end{lstlisting}

If a variable occurs in both events, an additional equality constraint
is synthesized.  For example,
\begin{lstlisting}
  # ts1 @ f(x) returns z => ts1 @ g(z) returns c
\end{lstlisting}
requires that both function @f@ and @g@ happen in the same transaction,
and the return value of @f@ is the same as the argument of @g@.
It will be translated to the core syntax as the following:
\begin{lstlisting}
  # ts1 @ f(x) returns z => ts2 @ g(y) returns c /\ ts1 == ts2 && x == y
\end{lstlisting}

It is totally fine to omit the time-stamp, arguments, or return values.
Simply requiring @f@ happens before @g@:
\begin{lstlisting}
  # f => g
  function f(...) { ... }
\end{lstlisting}

\todo{need to think about quantification.}

Note: @f => g@ matches the most recent call of @f@,
or any prior call of @f@ (if there are multiple calls of @f@)?

\todo{need to think about interactions with compiler optimization (eg inlining).}

\todo{need to think about how temporal contracts interact with address calls.}

\subsubsection{Negative Temporal Properties}

Negative temporal properties enforces that an event must \emph{not}
happen after another event.

\begin{lstlisting}
  # ts1 @ f(x) returns z =/> ts2 @ g(a) returns c /\ side-cond
\end{lstlisting}
It enforces that when @g@ is invoked, we check that @f@ has not invoked
\emph{and} the side condition is evaluated to true.
We do not enforce that @g@ cannot be invoked eventually
after invoking @f@.

For example, the following spec enforces that @g@ cannot be
invoked after @f@ using @z@ as the argument:
\begin{lstlisting}
  # ts1 @ f(x) returns z =/> ts2 @ g(z) returns c
  //or equivalently
  # ts1 @ f(x) returns z =/> ts2 @ g(y) returns c /\ x == y
\end{lstlisting}
However it is okay to call @g@ after @f@ with an argument different from @z@.

\subsubsection*{Affine Use}

Using negative temporal contracts, we can enforce affine use, i.e.
a function can be called at most once.
\begin{lstlisting}
  # f =/> f
  function f(int x) { ... }
\end{lstlisting}
If @f@ has been called before (and the call is finished), any
call to @f@ afterwards is prohibited.

\note{could further discern different timestamps/senders}

%linear (must call once),
%relevant (at least call once)

\subsection{Contextual Properties}

Contextual contracts inspect the calling-context/stack within the
current smart contract.

\note{Related work: stack inspection. But we don't need to modify the runtime
	of EVM, but simply record a shadow stack for necessary metadata. Is that enough?}

\subsubsection{Positive Contextual Properties}

For example, the following spec enforces that @g@ can only be invoked
under the calling context of @h@, and the return value of @h@ must be
the same as the argument of @g@:

\begin{lstlisting}
  function h(int x) returns int { ... }
  # h(x) returns y ~> g(y) returns r
  function g(int y) returns int { ... }
\end{lstlisting}

To enforce this behavioral contracts, we must check
\begin{itemize}
	\item when @g@ is invoked, there is a frame of @h@ on the stack,
	\item when @h@ returns (which happens after @g@ returns), the return value
	      is the same as the argument of @g@.
\end{itemize}

Question: what if @g@ is invoked multiple times under the same calling context of
@h@?

Solution: need quantifier. A simple/default solution is to check all
invocations of @g@.

Question: what if there are multiple call frames of @h@ on the stack?

Solution: need quantifier. A simple/default solution is to only check
the most recent call frame of @h@ on the stack.

\wac{I think the previous definition of contextual property $h ~> g$ means $\forall g, \exists h$ implicitly, i.e., for any $g$ called, there must exist one $h$ on the call stack. I cannot think of a use scenario where quantifiers on $h$ and $g$ are needed in practice.}
\wac{If we want to distinguish different call frames of $h$ and $g$, maybe it's better to design activation constraints for the contextual properties, e.g., \texttt{cond -> h ~> g}, meaning $h ~> g$ is only enforced when $cond$ is true (imply). ps: the special arrow is broken outside the listing environment. :(}

\subsubsection{Negative Contextual Properties}

The following program is a violation of non-reentrancy:

\begin{lstlisting}
  # non-reentrant
  function f(uint x) { f(n) }
\end{lstlisting}

This is an example of negative contextual properties, i.e. something cannot
happen under the current calling context. Under the neath, this spec is
equivalent to

\begin{lstlisting}
  # f ~/> f
  function f(uint x) { f(n) }
\end{lstlisting}
