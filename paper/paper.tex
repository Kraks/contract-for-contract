% -*- coding: utf-8; -*-
% vim: set fileencoding=utf-8 :
\documentclass[english,submission]{programming}
%% First parameter: the language is 'english'.
%% Second parameter: use 'submission' for initial submission, remove it for camera-ready (see 5.1)

\usepackage[backend=biber]{biblatex}
\addbibresource{references.bib}

%
% Packages and Commands specific to article (see 3)
%
% These ones  are used in the guide, replace with your own.

%% Some recommended packages.
\usepackage{booktabs}   %% For formal tables:
%% http://ctan.org/pkg/booktabs
\usepackage{subcaption} %% For complex figures with subfigures/subcaptions
%% http://ctan.org/pkg/subcaption

%\usepackage[uft8]{inputenc}
%\usepackage[T1]{fontenc}
%\usepackage[scaled=0.8]{beramono}
\usepackage{amsmath}
\usepackage{color}
\usepackage{xcolor,colortbl}
\usepackage{url}
\usepackage{listings}
\usepackage{paralist}
%\usepackage[compact]{titlesec}
\usepackage{caption}
\usepackage{wrapfig}
\usepackage{enumitem}
\usepackage{multicol}
\usepackage{flushend}
\usepackage{textcomp}
\usepackage{pdfpages}
%\usepackage{cleveref}
\usepackage{xspace}
\definecolor{light-gray}{gray}{0.85}

% ----- listings

%\definecolor{ckeyword}{HTML}{7F0055}
\definecolor{ckeyword}{HTML}{000000}
\definecolor{ccomment}{HTML}{3F7F5F}
\definecolor{cstring}{HTML}{2A0099}

\lstdefinelanguage{Solidity}%
{morekeywords={
  abstract, contracts, is,
  function, constructor, memory,
  for, if, new
  mapping,
  internal, external, pure, view, payable, return, returns,
  public, private, virtual, override,
  true, false, bool,
  bytes, bytes32,
  uint, uint8, uint16, uint32, uint64, uint128, uint256,
  address, string,
  require, assert, revert, try, catch
  },%
  sensitive,%
  moredelim=*[il][\bfseries]{\#\#\ },
  morecomment=[l]//,%
  morecomment=[s]{/*}{*/},%
  morestring=[b]",%
  %morestring=[b]',%
  showstringspaces=false%
}[keywords,comments,strings]%

\lstset{
  language=Solidity,%
  mathescape=true,%
%  columns=[c]fixed,%
  aboveskip=2pt,%\smallskipamount,
  belowskip=1pt,%\negsmallskipamount,
  lineskip=-1pt,
  basewidth={0.6em, 0.5em},%
%  backgroundcolor=\color{listingbg},
  basicstyle=\ttfamily,
  keywordstyle=\keywordstyle,
  commentstyle=\commentstyle,
  stringstyle=\stringstyle,
%  xleftmargin=0.5cm
  literate={-->}{{$\to$}}2
           {->}{{$\mapsto$}}3
           {<-}{{$\leftarrow$}}2
           {=>}{{$\Rightarrow ~$}}2
           {=/>}{{$\not\Rightarrow ~$}}3
           {~>}{{$\leadsto$}}2
           {~/>}{{$\not\leadsto$}}2
           {|-}{{$\ts$}}2
           {σ}{{$\sigma$}}1
           {ρ}{{$\rho$}}1
           {→}{{$\to$}}1
           {←}{{$\leftarrow$}}1
           {λ}{{$\lambda$}}1
           {α}{{$\alpha$}}1
           {⊔}{{$\sqcup$}}1
           {⊓}{{$\sqcap$}}1
           {⊑}{{$\sqsubseteq$}}1
           {⊤}{{$\top$}}1
           {⊥}{{$\bot$}}1
           {×}{{$\times$}}1
           {τ}{{$\tau$}}1
           {ψ}{{$\psi$}}1
           {Σ}{{$\Sigma$}}1
           {⟨}{{$\langle$}}1
           {⟩}{{$\rangle$}}1
           {π}{{$\pi$}}1
           {∪}{{$\cup$}}2
           {/\\}{{$\land$}}2
           %{[[}{{$[\![$}}1
           %{]]}{{$]\!]$}}1
           %{…}{{$\!...$}}1
}

\definecolor{listingbg}{RGB}{240, 240, 240}

\newcommand{\commentstyle}[1]{\color{ccomment}\itshape{#1}}
\newcommand{\keywordstyle}[1]{\color{ckeyword}\bfseries{#1}}
\newcommand{\stringstyle}[1]{\color{cstring}\text{#1}}

\lstnewenvironment{listing}{\lstset{language=Solidity}}{}
\lstnewenvironment{listingtiny}{\lstset{language=Solidity,basicstyle=\scriptsize\ttfamily}}{}

\newcommand{\code}[1]{\lstinline[language=Solidity,columns=fixed,basicstyle=\ttfamily]|#1|}

% ----- formal

\newcommand{\judgement}[2]{{\bf #1} \hfill #2}
\newcommand{\den}[1]{\llbracket~#1~\rrbracket}

% ----- comments and todo

\newcommand{\note}[1]{{\color{red}[#1]}}
\newcommand{\todo}[1]{\note{TODO: #1}}
\newcommand{\rev}[1]{\note{Revision: #1}}
\newcommand{\silent}[1]{}
\newcommand{\hl}[1]{\setlength{\fboxsep}{0pt}\colorbox{light-gray}{#1}}

\newcommand{\lang}{\textsf{ConSol}\xspace}

\newcommand{\lb}{\{~}
\newcommand{\rb}{~\}}
\newcommand{\la}{\langle}
\newcommand{\ra}{\rangle}

\newcommand{\Typ}[1]{\ensuremath{\mathsf{#1}}}
\newcommand{\Keywd}[1]{\ensuremath{\texttt{#1}}}
\newcommand{\Ast}[1]{\ensuremath{\mathsf{#1}}}
\newcommand{\Def}[1]{\ensuremath{\mathsf{#1}}}


\newcommand*{\CTAN}[1]{\href{http://ctan.org/tex-archive/#1}{\nolinkurl{CTAN:#1}}}
%%

%%%%%%%%%%%%%%%%%%
%% These data MUST be filled for your submission. (see 5.3)
\paperdetails{
  %% perspective options are: art, sciencetheoretical, scienceempirical, engineering.
  %% Choose exactly the one that best describes this work. (see 2.1)
  perspective=art,
  %% State one or more areas, separated by a comma. (see 2.2)
  %% Please see list of areas in http://programming-journal.org/cfp/
  %% The list is open-ended, so use other areas if yours is/are not listed.
  area={Social Coding, General-purpose programming},
  %% You may choose the license for your paper (see 3.)
  %% License options include: cc-by (default), cc-by-nc
  % license=cc-by,
}
%%%%%%%%%%%%%%%%%%

%%%%%%%%%%%%%%%%%%
%% These data are provided by the editors. May be left out on submission.
%\paperdetails{
%  submitted=2016-08-10,
%  published=2016-10-11,
%  year=2016,
%  volume=1,
%  issue=1,
%  articlenumber=1,
%}
%%%%%%%%%%%%%%%%%%


\begin{document}

\title{Contracts for Contracts}

\subtitle{Securing Smart Contracts with Higher-Order Temporal Behavioral Contracts}% optional

\titlerunning{Contracts for Contracts} %optional, in case that the title is too long; the running title should fit into the top page column

\author[a]{Tobias Pape}
\authorinfo{is the author of this {LaTeX} class. Contact him at
  \email{tobias.pape@hpi.uni-potsdam.de}.}
\affiliation[a]{Hasso Plattner Institute, University of Potsdam, Germany}

\author{Cristina V. Lopes}
\authorinfo{is associate editor for the first two issues of The Art, Science,
  and Engineering of Programming. Contact her at \email{lopes@ics.uci.edu}.}
\affiliation{University of California, Irvine, USA}

\author[a]{Robert Hirschfeld}
\authorinfo{is chair of the AOSA steering committee. The Art, Science,
  and Engineering of Programming is published by AOSA. Contact Robert at \email{hirschfeld@hpi.uni-potsdam.de}.}

% \authorrunning{T. Pape, C. Lopes, R. Hirschfeld} % Optional, for long author lists

\keywords{programming journal, paper formatting, submission preparation} % please provide 1--5 keywords


%%%%%%%%%%%%%%%%%%%%%%%%%%%%%
% Please go to https://dl.acm.org/ccs/ccs.cfm and generate your Classification
% System [view CCS TeX Code] stanz and copy _all of it_ to this place.
%% From HERE
\begin{CCSXML}
<ccs2012>
<concept>
<concept_id>10002944.10011122.10003459</concept_id>
<concept_desc>General and reference~Computing standards, RFCs and guidelines</concept_desc>
<concept_significance>300</concept_significance>
</concept>
<concept>
<concept_id>10010405.10010476.10010477</concept_id>
<concept_desc>Applied computing~Publishing</concept_desc>
<concept_significance>300</concept_significance>
</concept>
</ccs2012>
\end{CCSXML}

\ccsdesc[300]{General and reference~Computing standards, RFCs and guidelines}
\ccsdesc[500]{Applied computing~Publishing}

% To HERE
%%%%%%%%%%%%%%%%%%%%%%%

\maketitle

% Please always include the abstract.
% The abstract MUST be written according to the directives stated in
% http://programming-journal.org/submission/
% Failure to adhere to the abstract directives may result in the paper
% being returned to the authors.
\begin{abstract}
  abstract
\end{abstract}

\section{Introduction} \label{sec:intro}

%https://se.inf.ethz.ch/~meyer/publications/old/dbc_chapter.pdf
%https://docs.soliditylang.org/en/v0.8.17/control-structures.html?highlight=require#panic-via-assert-and-error-via-require

%\todo{why? solidity is rather flexible, unsatisfying/weaker type system,
%not all invariants can be checked statically, e.g. the status of the chain}
%One \note{common} use case of higher-order functions in smart contracts is
%to express callback \todo{explain how it is used}.

%\todo{it is hard to enforce properties when callbacks exist}
%Static verification cannot completely address the issue here since
%transaction developers are oblivious to the user-provided callback.

%Building trustworthy and secure smart contract programs requires solid
%understanding and reliable enforcement of program invariants.

Smart contracts are programs to automate the execution of transactional agreements so that all
parties involved in the agreement can be certain about the outcome.
Smart contract programs run on blockchains, implementing a consensus
protocol of Byzantine-fault-tolerant distributed ledgers.
By using blockchains, smart contracts are ``smart'' since they are an
mechanism to \emph{ensure the economic outcome of transactions} without a
central authority.

It is however ironic that despite the potential of popular smart contract
\emph{languages}, they often lack adequate mechanisms to ensure the
computational outcomes.  This unfortunate reality results in unsatisfactory
quality of smart contract programs. Without proper means to ensure the quality
of these programs, the aspiration to ensure the economic outcome of
transactions remains nothing more than a mere wish.

Undoubtedly, the past few years have witnessed several catastrophic failures
of blockchain systems resulting from attacks on vulnerable contract programs.
One prominent instance is the well-known DAO hack~\cite{DAOhack} where an unauthorized
attacker exploited the recursive function calls and callback mechanism, leading
to a staggering loss of \textdollar50 million. The fundamental issue at the core of this
attack was the failure of the deployed contract to prevent reentrancy under
critical conditions.

%\zz{ On the other hand, assertion primitives are crucial for ensuring the
%accuracy and integrity of a smart contract's business model. }
%\zz{ I'm not sure if this statement applies to the PL community, but from my
%understanding, assertions are typically used during testing rather than
%runtime.  However, in smart contracts, these assertions function as guard
%conditions and can cause reverts if violated, making them extremely important. }

%Solidity also provides the @modifier@ mechanism that intercepts
%before and after a function call, thus can be used to check
%pre- and post-conditions.
%\todo{can we compose modifier?}
%\todo{can we attach modifier to callbacks?}
%\todo{can we use modifier to check function's argument?}

To prevent such attacks, the programmer needs to specify those critical
conditions. The execution is allowed only when those conditions are met.
However, we observe that existing popular smart contract languages (e.g.,
Solidity) do not provide \emph{expressive, effective, convenient} means to
\emph{specify} and \emph{enforce} contract behaviors.
Simple first-order assertion primitives are indeed available, but they cannot
effectively examine callable addresses or function values,
%\zz{are function values referring to high-order functions?}
which exhibit
higher-order behaviors and are a common pattern used in developing smart contract
programs.
%Moreover, many of the critical conditions in smart contracts are \emph{temporal}
%-- the order of events matters.
%For instance, to defense the DAO attack caused by reentrancy via
%external functions, it requires to check that the function of interest cannot
%be invoked \emph{before} the current call of the same function is finished.
The lack of sufficient linguistic means to specify and enforce subtle but
critical behaviors discourages programmers to write clean and maintainable
code with higher-level abstractions.
Instead, programmers are obliged to write verbose low-level code
to implement security checks, which conflates the main business logic with
behavioral checks, further leading to poor readability and maintainability.
%The pervasive temporal conditions in distributed blockchain systems further
%complicates the situation: these temporal behavioral checks are usually
%encoded as value-level checks in ad-hoc manners.
%Taking the DAO attack as example again, a common way to implement ``reentrancy
%guards'' is to manually define an additional boolean flag, indicating
%whether the function is finished. This is both verbose and error-prone,
%quickly becoming unmanageable for complex behaviors.
And often worse, defensive checks are forgotten or disregarded, leading to
vulnerable code causing real monetary loss.
\todo{need a concrete motivating example?}

%Apart from the investigating effort to certify and verify smart contract
%programs, language designers are responsible to empower programmers with better
%linguistic solutions to secure smart contracts in the first place.
To address these issues, we argue that \emph{behavioral software
contracts}~\cite{DBLP:conf/tools/Meyer98a} should play a fundamental role in
the development of \emph{smart contract} programs.
Similar to smart contracts, behavioral contracts is an agreement, too. They
specify assumptions and guarantees between software components, just as
smart contracts specify assumptions and guarantees between economic parties.

Behavioral contracts are appealing to programmers since the specified
assumptions and guarantees are executable specifications written in the same
programming language syntax, and will be monitored and reported if violated.
Pioneered by the Eiffel language \cite{DBLP:books/ph/Meyer91} and the
design-by-contract methodology \cite{DBLP:conf/tools/Meyer98a,
DBLP:books/ph/Meyer97}, various styles of behavioral contracts have been
adopted in Java, Haskell, C\#, Python, Racket, Elixir
\cite{DBLP:conf/erlang/0001BBHMEF22}, etc. \note{cite}
Studies \cite{DBLP:conf/rodin/Chalin06, DBLP:books/ph/Meyer97}
have shown that behavioral contracts can effectively support design, development,
testing, and debugging of software systems.

In this paper, we develop the first behavioral contract system \lang (Contract
Solidity) for the Solidity language, providing a practical specification and
monitoring system for both first-order and higher-order behaviors.
%monitoring system for both \emph{higher-order and temporal behaviors}.
Solidity is a programming language to build smart contract programs that run on
the Ethereum platform.  As one of the most popular ecosystems of smart
contracts, Solidity has been used \todo{to do cool things}.
\todo{explain challenges}

%\lang enriches the Solidity language with \emph{higher-order} and
%\emph{temporal} behavioral contracts, which are absent in existing approaches.
%With higher-order and temporal behavioral specification, many downstream
%tasks become further amenable, including but not limited to static verification,
%guided testing, or specification extraction.

%However, Solidity and its running platform Ethereum is a complex system.
%Carelessly written contracts can cause serious bugs and loses.
%We identify \todo{solidity's tricky feature}

\subsection*{\textbf{Contracts for Address}}

One type of value in Solidity that exhibits higher-order behaviors is \emph{addresses}.

capability

\todo{restrict possible call targets}
\todo{why static verification fails}

\subsection*{\textbf{Monitoring Extent}}

extent

%\subsection*{\textbf{Contracts for Temporal Properties}}

%Temporal contracts provides a way to \todo{...}.

\subsection*{\textbf{Evaluation}}

Compared to existing approaches \todo{of what}, our work is both expressive,
easy-to-use, lightweight, effective, and economical.
Programmers can use all Solidity expressions to define the desired property,
which are attached only to relevant functions in a non-intrusive way.

We demonstrate that \lang is an effective approach to improve the robustness
and quality of smart contract programs.
By examining several cases of existing vulnerabilities, we show that our
approach is effective in preventing these attacks.
\zz{Let's mention the amount of funds we could save if we can prevent these attacks.}
We also evaluate the overhead (gas consumption) of our approach compared to
manually inserted assertion checks, showing that \todo{...}

\todo{why static verification fails}

%We develop a core model of \lang, including its syntax, operational semantics,
%and discuss its soundness guarantee.

%\todo{discuss performance/gas consumption}

%\todo{allows interesting programming idioms to be expressed}

% second-class contracts

\paragraph{Contributions} This paper makes the following contributions:
\begin{itemize}
  \item We design \lang, the first extension for Solidity that allows programmers to
        specify and enforce higher-order behaviors in smart contracts.
        We demonstrate \lang's use cases through extensive examples and
        implement \lang as a preprocessor for ordinary Solidity programs.
  \item We present the core formalization \lang, including its syntax, formal
        semantics, compilation, and soundness.
  \item We examine a number of real-world attacks with \lang, showing that our
        approach is effective in improving safety and reliability of smart contract
        programs.
  \item We evaluate the overhead of \lang compared to manually implemented
        assertions and checks, showing that our approach does not introduce
        additional runtime overhead.
\end{itemize}

\lang is open-sourced and publicly available at \note{url}.


\section{\lang by Examples}

We now introduce \lang informally with examples.
\lang is a non-breaking extension that additionally provides means
to specify behavioral contracts on top of the Solidity language.
We often use (executable) \emph{specification} for behavioral contracts to avoid terminological ambiguity.

\paragraph{Design Rationale}
This paper concerns about designing an extension for Solidity that enhances the
language with contracts for higher-order temporal behaviors.
The extension should be
\begin{itemize}
  \item non-intrusive
  \item expressive
  \item gradual (you do not need to pay the overhead if you don't use contracts)
  \item efficient (no or little overhead)
\end{itemize}

\subsection{Contracts for First-Order Values}

We begin with specifying precondition and postcondition for functions
involving first-order values. Consider the following \code{toEither} example
that converts \note{Wei?} to Either:
\begin{lstlisting}[language=Solidity]
# toEither(x) returns (y)
# requires x % 1e18 < 1e17
# ensures  y < type(uint).max / 1e18
function toEther(uint x) returns (uint y) { ... }
\end{lstlisting}

The first line introduces bindings for \code{toEither}'s arguments and return
values, which are \code{x} and \code{y}, respectively.
The binding names in the specification do not have to match those in the function definition.
The \code{requires}-clause specifies the precondition to call \code{toEither},
and the \code{ensures}-clause specifies the postcondition.
Any occurrences of \code{x} in the \code{requires}-clause refer to the actual
argument value, and similarly, occurrences of \code{y} in the \code{ensures}-clause
refer to the actual returned value.

\paragraph{Syntactic Sugars}
We can liberally omit any part of the argument or return value specification:
\begin{lstlisting}
# toWei(x) returns (_)
# ensures x < type(uint).max / 1e18
function toWei(uint x) returns (uint) { ... }
\end{lstlisting}
which is equivalent to core form where the omitted part is simply the boolean expression
\code{true}:
\begin{lstlisting}[language=Solidity]
# toWei(x) returns (_)
# ensures x < type(uint).max / 1e18
# requires true
\end{lstlisting}

\paragraph{Dependent Contract}
Postconditions can depend on the argument as well. In other words,
the scope of arguments span over both the precondition and postcondition.
For example, the following snippet is a valid postcondition that requires
monotonicity.
\begin{lstlisting}
# f(x) returns (y) ensures y > x
\end{lstlisting}

%Another way is to use the `any` flat contract, which is defined as
%\code{any = { _ | true }}:
%\begin{lstlisting}
%  # {x | x < type(uint).max / 1e18} -> any
%\end{lstlisting}

\paragraph{Any Expression is Allowed}
When specifying the pre-/postconditions of functions, the programmer can use
any valid Solidity expression to from the condition, including but not limited
to function calls, memory operations, and built-in special variables carrying
important transaction data (e.g. \code{msg.value}) or metadata (e.g.
\code{block.timestamp}) that are only available at runtime.
The programmers are free to check desired conditions against with these
transactional information.
For example, the following snippet examines \code{msg.value} as the
precondition of a payable function:
\begin{lstlisting}
# byTickets(n) ensures msg.value >= 1e15 * n
function buyTickets(int n) payable { ... }
\end{lstlisting}
With this kind of flexibility and expressiveness, programmers are able to
express invariants beyond the capability of static verification.
%\zz{How about other special variables, such as `block.chainid` and `tx.origin`?. Perhaps we could redesign a universal syntax for these variables in Section 2.5, where we discuss where to add the message sender}
%\url{https://docs.soliditylang.org/en/develop/units-and-global-variables.html#block-and-transaction-properties}

So far, these flat contracts for first-order values are straightforward,
and can be directly mapped to assertions wrapping around functions.

\subsection{Contracts for Higher-Order Values}

\paragraph{Addresses}
Resembling pointers in C/C++, addresses in Solidity are numerical values
that can be compared or \note{computed?}.
In this sense, specifications of addresses are no different
from other integer values.
For instance, the following specification requires that
the address argument value cannot be 0:
\begin{lstlisting}
# f(addr) requires addr == 0
function f(address payable addr) { ... }
\end{lstlisting}

However, addresses in Solidity have rich higher-order behaviors: they may
represent a deployed external Ethereum contract and can be invoked in
customized ways.
For example, the following snippet shows a function \code{f} taking an address
\code{addr} that is called at some point.
The programmer intends to the specify the precondition and postcondition when
calling \code{addr}. \note{GW: explain abi?}
Nevertheless, such checks against \code{addr} cannot be performed when \code{f}
is called, even though \code{addr} is a proper argument of \code{f}.
Because \code{addr} in this use case is effectively a function, and
it is undecidable to statically examine its properties in general.
Additionally, we may not have access to the source code of the contract
represented by \code{addr}.
\note{GW: could use a simpler example, and discuss low-level call in the next paragraph section}
\begin{lstlisting}
# f(addr) where
# { addr.call(msg, amount) returns (flag, data)
#   requires amount > 100
#   ensures flag == true }
function f(address payable addr) {
  ... // omitting statements before the call
  (bool success, bytes memory data) =
    addr.call(abi.encodeWithSignature(
        "foo(string, uint256)", "call foo", 128
      )
    );
  ... // omitting statements after the call
}
\end{lstlisting}
\lang addresses this issue by adopting lazy checks
\cite{DBLP:conf/icfp/FindlerF02} that only take place
when the address is actually invoked \note{GW: fwd ref to impl}.
The above snippet specifies the condition of \code{addr}
in the \code{where}-clause, which requires
the \code{amount} argument must be greater than 100 and
the address call must succeed.

Notably, using \lang we have separated the concern of specifications and usages
--- there is no need to change the function body or manually insert checks
around the call. This leads to more maintainable code and allows programmers to
fearlessly refactor the code.

\paragraph{Single Address, Multiple Callees}
Multiple callees can inhabit in a single address value.
Other than the \code{call} method, there are other forms of low-level calls with
different signatures, e.g., \code{delegatecall}, \code{send}, etc.
% https://docs.soliditylang.org/en/v0.8.20/units-and-global-variables.html#address-related
Programmers can specify conditions for different callee targets in the
\code{where}-clause.
\begin{lstlisting}
# f(addr) where
# { addr.send(x) returns (succ) requires ... }
# { addr.call(msg, x) returns (succ, data) requires ... }
\end{lstlisting}

\zz{I guess we should educate the audience about the difference between low-level and high-level calls at the beginning of Section 2.4. I'll take care of this later.}
\note{GW: it seems we need to discern \code{call}s with different signatures}

\paragraph{Additional Arguments}
Solidity address calls can take additional special arguments
such as \code{value} and \code{gas}:
\begin{lstlisting}
addr.call{value: msg.value, gas: 5000}(...);
\end{lstlisting}
In \lang, programmers can specify their conditions by introducing additional
bindings using the familiar syntax:
\begin{lstlisting}
# addr.call{value: v, gas: g}(msg, amount) ...
\end{lstlisting}
\code{v} and \code{g} represent the actual message value and gas
value, which can be used in \code{requires}/\code{ensures}-clauses.

\iffalse
It would be convenient to directly enforces the success of the call and omit
the data:
\begin{lstlisting}
  # { a | { {value, gas}(arg) | <pre-cond> } -> { (true, _) | true } }
  function f(address a) { ... }
\end{lstlisting}

\zz{just FYI, `a.f` can also specify `value`, `gas`, etc.}

\begin{lstlisting}
  # { a.f | pre-cond -> post-cond }
  # { a.g | pre-cond -> post-cond }
  function f(address a) { ... }
\end{lstlisting}

\wac{We may need an example here since the necessity to designate the callee function may not be obvious to reviewers.}
\fi

\paragraph{Second-Class Monitoring}
%\note{GW: an easy option is to discard enforcement of escaped addresses at all.
%But it seems we can guard escaped addresses within the current contract scope.
%We need to think about layered/stacked contracts of addresses (lax/picky).}
Addresses, as first-class values, can of course escape from a function, e.g.
by returning from the function or storing in external data structures.
However, it is not always possible to enforce the specification
of addresses when escaped to third-parties.
For example, consider a function that returns an address equipped with
conditions:
\begin{lstlisting}
EXAMPLE TODO
function f(address a) returns (address) { return a; }
\end{lstlisting}
In this case, if the returned address value crosses the boundary to other
Ethereum programs (by external calls initiated by a third-party) that we do not
have control, we are impotence to monitor if the conditions are violated.

In spite of that, \lang does guarantee that addresses are completely monitored
within the current Ethereum program's calling context, following a stack
discipline.
In other words, for an argument addresses \code{addr} of function \code{f},
\lang is able to detect any violation of \code{addr}'s conditions before
\code{f} returns (even if the call happens in other functions).
We called this mechanism \emph{second-class monitoring}, borrowing Strachey's taxonomy
of first- and second-class objects \cite{DBLP:journals/lisp/Strachey00}.
This is already useful to prevent \note{XXX} \note{fwd ref}, especially when
the address comes from an untrusted party.

Similarly, when a guarded address is used as an argument to another address
call, \note{we lose the control of ...}

\paragraph{Notes on Function Pointers}
Likewise, functions in Solidity are first-class values, i.e., they can be used
as arguments of functions and returned from functions.
However, at this moment \footnote{Solidity version 8.20} there is no way to
write anonymous functions (lambda expressions) or nested, named functions that
capture variables from environments.
Only explicitly defined top-level functions can be used in ``first-class'' ways.
This restriction prohibits real closure values following lexical scopes,
familiarized by functional programmers.
%This restriction discourages programmers to use higher-order functions due to
%its inconvenience and verbosity.
Moreover, it poses challenges to seamlessly implement first-class behavioral
contract and monitor system that guards higher-order functions (e.g. as in
\cite{DBLP:conf/icfp/FindlerF02}) without an expensive whole program
transformation.
Therefore, in this paper we focus on the contract and monitoring of addresses
values and leave monitoring for higher-order functions as future work when
Solidity has a proper support for lambda expressions.

\iffalse
	\begin{lstlisting}
function map(uint[] memory data, function (uint) pure returns (uint) f)
  internal pure returns (uint[] memory r)
{
  r = new uint[](data.length);
  for (uint i = 0; i < data.length; i++) {
    r[i] = f(data[i]);
  }
}
\end{lstlisting}

We can specify the contract for function arguments too:
\begin{lstlisting}
  # { f | {x | x < 0} -> {y | y > 0} }
  function map(int[] memory data, function (int) pure returns (int) f) { ... }
\end{lstlisting}
It might be too verbose -- so we can define those predicates separately for better readability/maintainability:
\begin{lstlisting}
  function greaterThanZero(int x) pure returns (bool) {
    return x > 0;
  }
  function smallerThanZero(int x) pure returns (bool) {
    return x < 0;
  }
  # { f | smallerThanZero -> greaterThanZero }
  function map(int[] memory data, function (int) pure returns (int) f) { ... }
\end{lstlisting}

Functions contracts can be higher-order --
it can take other function contracts
as part of the spec. For example
\zz{coooool!}
\begin{lstlisting}
  # TODO
  function f(function (function (int) returns (int) g) h) { ... }
\end{lstlisting}

Function contracts are first-class -- so if this guarded function is escaped
(e.g. by returning), the contract is still enforced:
\begin{lstlisting}
NEED AN EXAMPLE
\end{lstlisting}
\fi

%%% Local Variables:
%%% mode: latex
%%% TeX-master: "paper"
%%% End:

\section{Formal Model}
\label{sec:model}

the core model of the language $\lambda_\lang$

\subsection{Syntax}

\begin{figure}
  \begin{alignat*}{3}
    \Typ{Variables}  \qquad & x,&&y,z,f,g \\
    \Typ{Types}      \qquad & t && ::=
      \Keywd{address} \mid
      \Keywd{uint} \mid
      \Keywd{function}(t, \dots)~\optional{\Keywd{returns}~t} \mid
      \dots \\
    \Typ{Literals}    \qquad & v && \in
      \mathbb{Z} \mid
      \todo{whatelse} \\
    \Typ{Operators} \qquad & o && ::=
      + \mid
      - \mid
      \times \mid
      \land \mid
      \lor \mid
      \dots \\
    \Typ{Expressions}\qquad & e && ::=
      v \mid
      x \mid
      o(e, \dots) \mid
      \todo{} \\
    \Typ{Statements} \qquad & s && ::=
                                   \Keywd{revert} \mid
                                   s_1; s_2 \mid
                                   x ~\mathsf{:=}~ e \mid
                                   x ~\mathsf{:=}~ f(e, \dots)  \text{\yy{unify assignments?}}\\
                            & && \mid \Keywd{if}~e~\Keywd{then}~s_1~\optional{\Keywd{else}~s_2} \mid
                                   \Keywd{while}~e~\Keywd{do}~s
                                   \\
    \Typ{Temporal~Connectives} \qquad & \oplus && ::=
      ~ \Rightarrow ~\mid~ \not\Rightarrow ~\mid~ \leadsto ~\mid~ \not\leadsto \\
    \Typ{Value~Specifications} \qquad & \varsigma_v && ::=
      \{ (x, \dots) ~\vert~ e \} \mid
      \varsigma_v \mapsto \varsigma_v \\
    \Typ{Temporal~Specifications} \qquad & \varsigma_t && ::=
     f(x, \dots)~\Keywd{returns}~ (z_1, \dots) ~\oplus~ g(y, \dots)~\Keywd{returns}~ (z_2, \dots) \land e \\
    \Typ{Specifications} \qquad & \varsigma && ::= \varsigma_v \mid \varsigma_t \\
    \Typ{Function~Definitions} \qquad & \mathcal{F} && ::=
      \# \varsigma^* ~ \Keywd{function}~f(t ~ x, \dots)~\optional{\Keywd{returns}~t}~\{ ~s~ \}
  \end{alignat*}
  \caption{The abstract syntax of \lang.}
  \label{fig:syntax}
\end{figure}

%%% Local Variables:
%%% mode: latex
%%% TeX-master: "paper"
%%% End:


\subsection{Semantics}

static semantics

dynamic semantics
\yy{do we want to model the context of contracts? (instead of reasoning one contract)}

The semantics of contracts

\subsection{Compilation of Specification}

\subsection{Soundness}

what does it guarantee


\section{Effectiveness Evaluation} \label{sec:case}
\begin{itemize}
	\item Overall result in preventing several real-world attacks.
\end{itemize}

We now turn our attention to real world cases and examine how \lang
can be used to consolidate smart contract programs.
These case studies are empirical evidence showing \lang is effective
in preventing attacks.

\subsection{Reentrancy}
Market.xyz readonly reentrancy attack.
Transfer return value not checked.
Li.fi attack: Arbitrary call to untrusted code (low-level call).

\paragraph{Severity} talk about severity for each case

\subsection{Callback}

\subsection{Comparison with \texttt{modifier}}

\todo{what else?}



\section{Implementation}

We implement \lang as a preprocessor of Solidity programs annotated with
\lang's specifications (\todo{bwd ref}). Given a such input program, \lang analyzes where and
which condition needs to be inserted into the original program.
The output program can be compiled and deployed using downstream Solidity
toolchains without any further modification.

Our current implementation does not handle inlined assembly. \todo{why}

Contracts for higher-order functions is another future work when Solidity has a
proper support for lambda expressions. Without such proper support from the base language,
\todo{...}


\section{Performance Evaluation}
\label{sec:eval}
\begin{itemize}
	\item Gas cost
	\item DeFi projects with source code and test cases.
	\item Migrate ``Securing smart contract with runtime validation'' data (with ERCxxx invariant).
\end{itemize}

\section{Readability Evaluation}
\begin{itemize}
	\item Number hunk of code to measure the efforts to fix a bug.
	\item Number of tokens required to fix.
\end{itemize}

\section{Related Work}

\subsubsection*{\textbf{Behavioral Contracts}}

Eiffel the ``design-by-contract'' methodology \cite{DBLP:books/ph/Meyer91, DBLP:conf/tools/Meyer98a}

higher-order contract \cite{DBLP:conf/icfp/FindlerF02}

temporal higher-order contract \cite{DBLP:conf/icfp/DisneyFM11}

contracts for monitoring termination \cite{DBLP:conf/pldi/NguyenGTH19}, a form
of liveness properties, i.e.
something good will happen eventually.
This work only considers safety property, i.e. monitoring and preventing
something bad will not happen.

stack inspection \cite{DBLP:conf/popl/FournetG02, DBLP:conf/sp/WallachF98}

interface automata \cite{DBLP:conf/sigsoft/AlfaroH01}

\subsubsection*{\textbf{Smart Contracts}}

core calculus of Solidity-like languages \cite{Sergey2021, DBLP:conf/esorics/BartolettiGM19, DBLP:conf/fc/CrafaPZ19}

runtime validation, reduce overhead of runtime checks \cite{DBLP:conf/pldi/LiCL20}

security analyzer \cite{DBLP:conf/pldi/BrentGLSS20}

nondeterminisitic payment bugs \cite{DBLP:journals/pacmpl/WangZS19}

behavioral simulation to verify smart contracts \cite{DBLP:conf/pldi/BeillahiCEE20}

formal and modular specification of smart contracts \cite{DBLP:journals/pacmpl/BramEMSS21}

static analysis to infer ownership and commutativity summaries, for parallelism
\cite{DBLP:conf/pldi/Pirlea0S21}

callbacks \cite{DBLP:journals/pacmpl/AlbertGRRRS20, DBLP:journals/pacmpl/GrossmanAGMRSZ18}

reduce gas consumption \cite{DBLP:journals/pacmpl/GrechKJBSS18}

static analysis of Ethereum \cite{DBLP:journals/pacmpl/SmaragdakisGLTT21}

\input{conclusion}

\acks
We want to thank you for using this class to prepare articles for The Art, Science, and Engineering of Programming.

%\appendix
\printbibliography

\end{document}

% Local Variables:
% TeX-engine: luatex
% End:
