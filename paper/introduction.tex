\section{Introduction}
\label{sec:intro}

%https://se.inf.ethz.ch/~meyer/publications/old/dbc_chapter.pdf
%https://docs.soliditylang.org/en/v0.8.17/control-structures.html?highlight=require#panic-via-assert-and-error-via-require

%\todo{why? solidity is rather flexible, unsatisfying/weaker type system,
%not all invariants can be checked statically, e.g. the status of the chain}
%One \note{common} use case of higher-order functions in smart contracts is
%to express callback \todo{explain how it is used}.

%\todo{it is hard to enforce properties when callbacks exist}
%Static verification cannot completely address the issue here since
%transaction developers are oblivious to the user-provided callback.

%Building trustworthy and secure smart contract programs requires solid
%understanding and reliable enforcement of program invariants.

Smart contracts are programs to automate the execution of agreement so that all
parties involved in the agreement can be certain about the outcome.
Smart contracts are ``smart'' since by using blockchains they provide a
mechanism to \emph{ensure the economic outcome of transactions} without a
central authority.
A blockchain implements a consensus protocol of Byzantine-fault-tolerant
distributed ledgers.

It is however ironic that many popular smart contract languages, which hold the
great promise to ensure transactional outcomes via consensus and distributed
ledgers, do not provide enough means to \emph{ensure the computational outcomes}.
This unfortunate reality leads to poor quality of programs.
Without ensuring the quality of programs, it is impossible to ensure the
economic outcome of transactions.

Indeed, in recent years, we have seen a number of failures of smart contract
systems due to attacks utilizing vulnerable contract programs.
For example, in the famous DAO hack\note{cite}, the unauthorized attacker is
able transfer money by constructing recursive function calls with
the callback mechanism, causing a total lose of \$50 million.
The root cause of this attack is that the deployed contract does not
checks reentrancy defensively before performing critical actions.

%\zz{ On the other hand, assertion primitives are crucial for ensuring the
%accuracy and integrity of a smart contract's business model. }
%\zz{ I'm not sure if this statement applies to the PL community, but from my
%understanding, assertions are typically used during testing rather than
%runtime.  However, in smart contracts, these assertions function as guard
%conditions and can cause reverts if violated, making them extremely important. }

%Solidity also provides the @modifier@ mechanism that intercepts
%before and after a function call, thus can be used to check
%pre- and post-conditions.
%\todo{can we compose modifier?}
%\todo{can we attach modifier to callbacks?}
%\todo{can we use modifier to check function's argument?}

To prevent such attacks, the programmer needs to specify those critical
conditions. The execution is allowed when those conditions are met.
However, we observe that existing popular smart contract languages (e.g.
Solidity) do not provide \emph{rich, effective, convenient} means to specify
and enforce contract behaviors.
Simple first-order assertion primitives (e.g., @require@) are indeed available,
but they cannot effectively examine callable addresses or function values, which
are a common pattern used in smart contract programs.
Moreover, many of the critical condition in smart contracts are \emph{temporal}
-- the order of happened events matters.
For instance, to defense the DAO attack caused by reentrancy via
external functions, it requires to check that the function of interest cannot
be invoked \emph{before} the current call of the same function is finished.

The lack of sufficient linguistic means to specify and enforce subtle but
critical behaviors discourages programmers to use concise higher-level
abstractions when writing smart contract code.
Instead, to be secure, programmers are obliged to write verbose code with
lower-level or no abstractions, which conflates the main business logic with
behavioral checks, further leading to poor readability and maintainability.
Taking the DAO attack as example again, a common way to implement ``reentrancy
guards'' is to manually define an additional boolean state, indicating
if the function is finished.
And often worse, defensive checks can be easily forgotten or disregarded,
leading to vulnerable code causing real monetary lose.

%Apart from the investigating effort to certify and verify smart contract
%programs, language designers are responsible to empower programmers with better
%linguistic solutions to secure smart contracts in the first place.
In this paper, we argue that \emph{behavioral software contracts}
\cite{DBLP:conf/tools/Meyer98a} should play a fundamental role in the
development of \emph{smart contract} programs.  Similar to smart contracts,
behavioral contracts is an agreement too. They specify the assumptions and
guarantees between software components. The specified assumptions and
guarantees will be monitored and reported when violated in languages with
behavioral contracts.

Pioneered by the Eiffel language \cite{DBLP:books/ph/Meyer91} and the
design-by-contract methodology \cite{DBLP:conf/tools/Meyer98a}, various styles
of behavioral contracts have been widely adopted in Java, Haskell, C\#, Python,
Racket, Elixir \cite{DBLP:conf/erlang/0001BBHMEF22}, etc. \note{cite}
In this paper, we develop a behavioral contract system \lang for the Solidity
language, providing a practical specification and monitoring system
for both higher-order and temporal behaviors.
Solidity is a programming language to build smart contract programs that run on
the Ethereum platform.  As one of the most popular languages for smart
contracts, Solidity has been used \todo{to do cool things}.

We demonstrate that behavioral contracts are an effective approach
to improve the robustness and quality of smart contract programs.
\todo{summarize what we can prevent/achieve}

%\lang enriches the Solidity language with \emph{higher-order} and
%\emph{temporal} behavioral contracts, which are absent in existing approaches.
%With higher-order and temporal behavioral specification, many downstream
%tasks become further amenable, including but not limited to static verification,
%guided testing, or specification extraction.

%However, Solidity and its running platform Ethereum is a complex system.
%Carelessly written contracts can cause serious bugs and loses.
%We identify \todo{solidity's tricky feature}


\subsection*{\textbf{Contracts for Higher-Order Values}}

One type of values in Solidity that exhibits higher-order behaviors is \emph{addresses}.

Functions in Solidity are first-class too, meaning that functions can be used as
arguments for and returned from functions. However, at this moment
\footnote{Solidity version 8.20} Solidity has not supported writing anonymous
functions (lambda expressions) or nested (named) functions.
Only explicitly defined top-level functions can be used in first-class ways,
which discourages programmers to use higher-order functions due to its
inconvenience and verbosity.
Moreover, this restriction poses challenges to seamless implement first-class
behavioral contract and monitor system (e.g. as in \cite{DBLP:conf/icfp/FindlerF02})
without an expensive whole program transformation.
Therefore, in this paper we focus on the contract and monitoring of addresses values
and leave monitoring for higher-order functions as future work when
Solidity has a proper support for lambda expressions.

\todo{restrict possible call targets}
\todo{why static verification fails}

\subsection*{\textbf{Contracts for Temporal Properties}}

Temporal contracts provides a way to \todo{...}.
\todo{why static verification fails}

\paragraph{}

We develop a core model of \lang, including its syntax, operational semantics,
and discuss its soundness guarantee.
With this extension of higher-order temporal behavioral contracts, we examine
several cases of reported bugs and vulnerabilities and show that
our approach is effective in preventing these attacks.
\zz{Let's mention the amount of funds we could save if we can prevent these attacks.}
We also evaluate the overhead (gas consumption) of our approach
compared to manually inserted assertion checks, showing that \todo{...}

%\todo{discuss performance/gas consumption}

%\todo{allows interesting programming idioms to be expressed}

% second-class contracts

\subsubsection*{\textbf{Contributions}} This paper makes the following contributions:
\begin{itemize}
  \item We design \lang, an extension for Solidity that allows programmers to
        specify and enforce higher-order and temporal behaviors in smart contracts.
        We demonstrate \lang's use cases through extensive examples and
        implement \lang as a preprocessor for ordinary Solidity programs.
	\item We present the core formalization $\lambda_\lang$ for our proposed
	      extension, including its syntax, formal semantics, compilation, and
	      soundness.
  \item We examine a number of real-world attacks with \lang, showing that our
        approach is effective in improving safety and reliability of smart contract
        programs.
  \item We evaluate the overhead of \lang compared to manually implemented
        assertions and checks, showing that our approach does not introduce
        additional runtime overhead.
\end{itemize}
\lang is open-sourced and publicly available at \note{url}.

